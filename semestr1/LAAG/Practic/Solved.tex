\documentclass[12pt,a4paper]{article}
\usepackage[utf8]{inputenc}
\usepackage[T2A]{fontenc}
\usepackage[ukrainian]{babel}
\usepackage{fancyvrb}
\usepackage{pdflscape}

\usepackage{amsmath} % у преамбулі
\usepackage{array, multirow}
\usepackage{hyperref} % <-- Обов’язково підключіть цей пакет
\usepackage{caption}
\usepackage{booktabs}

\usepackage{xcolor}

\renewcommand{\thetable}{№\arabic{table}}
\captionsetup[table]{name=Таблиця}  % замість "Табл." буде "Таблиця"

\usepackage{graphicx} % <-- Для роботи з \includegraphics
\usepackage{geometry}
\geometry{
    left=2cm,
    right=2cm,
    top=2cm,
    bottom=2cm
}

\begin{document}

    \begin{titlepage}

        \thispagestyle{empty}
        \begin{center}
            \large
            Національний технічний університет України\\
            «Київський політехнічний інститут імені Ігоря Сікорського»\\[1em]
            Факультет інформатики та обчислювальної техніки\\
            Кафедра обчислювальної техніки
        \end{center}

        \vfill

        \begin{center}
            \textbf{\LARGE Лінійна алгебра та аналітична геометрія}\\[2em]
            \textbf{\Large Вирішення задач практикуму}\\[0.5em]
            (за збірником \textit{І. В. Алексєєвої, В. О. Гайдея,\\
            О. О. Диховичного, Л. Б. Федорової})
        \end{center}

        \vfill

        \begin{flushright}
            Виконав: студент 1 курсу ФІОТ, гр. ІО-41\\
            \textit{Давидчук А. М.}\\
        \end{flushright}

        \vfill

        \begin{center}
            Київ -- 2025
        \end{center}

    \end{titlepage}

    \noindent \textbf{№ 1.4:}

    Матриця 
    \( A = 
    \begin{pmatrix}
    a & b \\
    c & d \\
    e & f
    \end{pmatrix}
    \)
    має розмірність $3\times2$; $a_{21} = c, a_{32} = f$; Елемент $d$ знаходиться на позиції $a_{22}$, тобто має індекси: $i = 2, j = 2$.

    \vspace{1em}

    \noindent \textbf{№ 1.5:}

    \vspace{1em}
    1) Розмірність: $2\times3$; Рядки: \( a_{1j} = 
    \begin{pmatrix}
    4 & -7 & 5 
    \end{pmatrix}
    \), \( a_{2j} = 
    \begin{pmatrix}
    -6 & 8 & -1 
    \end{pmatrix}
    \); Стовпці: \( a_{i1} = 
    \begin{pmatrix}
    4 \\
    -6
    \end{pmatrix}
    \), \( a_{i2} = 
    \begin{pmatrix}
    -7 \\
    8
    \end{pmatrix}
    \), \( a_{i3} = 
    \begin{pmatrix}
    5 \\
    -1
    \end{pmatrix}
    \); $a_{23} = -1, a_{32}$ не існує.

    \vspace{1em}
    2) Розмірність: $2\times4$; Рядки: \( a_{1j} = 
    \begin{pmatrix}
    -6 & 4 & -1 & 0 
    \end{pmatrix}
    \), \( a_{2j} = 
    \begin{pmatrix}
    -9 & 0 & 1/2 & 2 
    \end{pmatrix}
    \); Стовпці: \( a_{i1} = 
    \begin{pmatrix}
    -6 \\
    -9
    \end{pmatrix}
    \), \( a_{i2} = 
    \begin{pmatrix}
    4 \\
    0
    \end{pmatrix}
    \), \( a_{i3} = 
    \begin{pmatrix}
    -1 \\
    1/2
    \end{pmatrix}
    \), \( a_{i4} = 
    \begin{pmatrix}
    0 \\
    2
    \end{pmatrix}
    \); $a_{23} = 1/2, a_{32}$ не існує.

    \vspace{1em}
    3) Розмірність: $3\times3$; Рядки: \( a_{1j} = 
    \begin{pmatrix}
    1 & -2 & 3 
    \end{pmatrix}
    \), \( a_{2j} = 
    \begin{pmatrix}
    -4 & 5 & -6
    \end{pmatrix}\), \( a_{3j} = 
    \begin{pmatrix}
    7 & -8 & 9
    \end{pmatrix}
    \); Стовпці: \( a_{i1} = 
    \begin{pmatrix}
    1 \\
    -4 \\
    7
    \end{pmatrix}
    \), \( a_{i2} = 
    \begin{pmatrix}
    -2 \\
    5 \\
    -8
    \end{pmatrix}
    \), \( a_{i3} = 
    \begin{pmatrix}
    3 \\
    -6 \\
    9
    \end{pmatrix}
    \)
    ; $a_{23} = -6, a_{32} = -8$.

    \vspace{1em}
    4) Розмірність: $3\times2$; Рядки: \( a_{1j} = 
    \begin{pmatrix}
    -1 & 0 
    \end{pmatrix}
    \), \( a_{2j} = 
    \begin{pmatrix}
    3 & 4
    \end{pmatrix}\), \( a_{3j} = 
    \begin{pmatrix}
    -7 & 5
    \end{pmatrix}
    \); Стовпці: \( a_{i1} = 
    \begin{pmatrix}
    -1 \\
    4 \\
    -7
    \end{pmatrix}
    \), \( a_{i2} = 
    \begin{pmatrix}
    0 \\
    4 \\
    5
    \end{pmatrix}
    \); $a_{23}$ не існує, $a_{32} = 5$.

    \vspace{1em}

    \noindent \textbf{№ 1.6:}

    Матриця $A$ --- квадратна порядку 2, матриця $C$ --- квадратна 3 порядку;
    Для матриці $A$ головна діагональ: \(
    \begin{pmatrix}
    3 & 0 
    \end{pmatrix}
    \), побічна діагональ:  \(
    \begin{pmatrix}
    -2 & 4 
    \end{pmatrix}
    \); Для матриці $C$ головна діагональ: \(
    \begin{pmatrix}
    c_{11} & c_{22} & c_{33} 
    \end{pmatrix}
    \), побічна діагональ:  \(
    \begin{pmatrix}
    c_{13} & c_{22} & c_{31} 
    \end{pmatrix}
    \).

    \vspace{1em}

    \noindent \textbf{№ 1.7:}

    Верхня трикутна матриця: $C$; Нижня трикутна матриця: $B$; Діагональна матриця: $A$.

    \vspace{1em}

    \noindent \textbf{№ 1.8:}

    1) Ні. 2) Ні. 3) Так. Одинична матриця 4-го порядку $E_{4\times4}$:
    \(
    \begin{pmatrix}
    1 & 0 & 0 & 0 \\
    0 & 1 & 0 & 0 \\
    0 & 0 & 1 & 0 \\
    0 & 0 & 0 & 1
    \end{pmatrix}
    \)

    \vspace{1em}

    \noindent \textbf{№ 1.9:}

    Матриця $A$: квадратна 3-го порядку;
    Матриця $B$: квадратна 2-го порядку, верхня трикутна;
    Матриця $C$: квадратна 3-го порядку, одинична, діагональна; 
    Матриця $D$: квадратна 3-го порядку;
    Матриця $F$: матриця-стовпець (вектор-стовпець);
    Матриця $G$: матриця-рядок (вектор-рядок).

    \vspace{1em}

    \noindent \textbf{№ 1.10:}

    1) $x = -3, y = 5$; 
    2) $x = -3/2, y = 5$; 
    3) $x = -1, y = 1$; 
    4) $x = 1, y = -5$; 
    5) $x = -1, y = 3, z = 4$;
    6) $x = 2, y = 9, z = 0$.

    \vspace{1em}

    \noindent \textbf{№ 1.11:}

    Ні; Так, нульову матрицю.

    \vspace{1em}

    \noindent \textbf{№ 1.12:}

    Для узгоджених (кількість стовпців першого множника дорівнює кільості рядків другого множника);
    Так, можна. Кожен елемент кожного рядка першого множника помножити на відповідний їм кожен елемент кожного стовпця другого множника
    (або інкаше кажучи, якщо взяти числа координати елемента добутку матриць $i, j$, то індекс $i$ вказуватиме на $i$ рядок першого множника, а
    індекс $j$ на $j$ стовпець другого множника, які потрібно почленно помножити, тобто $\displaystyle (AB)_{ij} = \sum_{k=1}^{n} a_{ik} \cdot b_{kj}$).

    \vspace{1em}

    \noindent \textbf{№ 1.13:}

    Ні, вони не узгоджені. Коли матриці $A$ та $B$ є квадратними одного порядку;
    Коли матриця $A$ є квадратною матрицею.

    \vspace{1em}

    \noindent \textbf{№ 1.14:}

    Не для всіх довільних матриць $A$ та $B$; Так, можлива.

    \vspace{1em}

    \noindent \textbf{№ 1.15:}

    1) Ні; 
    2) Так; 
    3) Так;
    4) Ні;
    5) Ні.

    \vspace{1em}

    \noindent \textbf{№ 1.16:}

    1) $m = 2, n = 3$; 
    2) $m = 3, n = 4$; 
    3) $m = 4, n = 3$; 
    4) $m = n = 2$; 
    5) $m = 9, n = 1$; 
    6) $m = 7, n = 6$; 
    7) $m = 2, n = 3$; 
    8) $m = 5, n = 4$.

    \vspace{1em}

    \noindent \textbf{№ 1.17:}

    Будуть такі: $n \times m$; Якщо елемент $a_{ij}$ це елемент матриці $A_{m\times n}$, 
    то в транспонованій матриці він стане $a_{ji}$.

    \vspace{1em}

    \noindent \textbf{№ 1.18:}

    Для кожної. $\displaystyle \left( A^T \right)^T = A$; Так, можуть, як приклад, симетричні матриці.

    \vspace{1em}

    \noindent \textbf{№ 1.19:}

    Будемо вважати, що $\vec{a}_i$ --- це $i$-тий стовпець певної матриці. Тоді
    $\vec{a}_1$ = \(
    \begin{pmatrix}
    1 \\
    2 
    \end{pmatrix}
    \), $\vec{a}_2$ = \(
    \begin{pmatrix}
    -1 \\
    0 
    \end{pmatrix}
    \), а також $\vec{b}_1$ = \(
    \begin{pmatrix}
    1 \\
    -2 
    \end{pmatrix}
    \), $\vec{b}_2$ = \(
    \begin{pmatrix}
    1 \\
    3 
    \end{pmatrix}
    \). 

    1) $\vec{a}_1 + \vec{a}_2 = $ \(
    \begin{pmatrix}
    1 \\
    2 
    \end{pmatrix} +
    \begin{pmatrix}
    -1 \\
    0 
    \end{pmatrix}
     = \begin{pmatrix}
    0 \\
    2 
    \end{pmatrix},
    \vec{a}_1 - \vec{a}_2 = \begin{pmatrix}
    1 \\
    2 
    \end{pmatrix} -
    \begin{pmatrix}
    -1 \\
    0 
    \end{pmatrix}
     = \begin{pmatrix}
    2 \\
    -2 
    \end{pmatrix}
    ,
    2\vec{a}_1 + 3\vec{a}_2 = 2\begin{pmatrix}
    1 \\
    2 
    \end{pmatrix}+
    \)

    \(
    +3\begin{pmatrix}
    -1 \\
    0 
    \end{pmatrix}
    =
    \begin{pmatrix}
    2 \\
    4 
    \end{pmatrix} +
    \begin{pmatrix}
    -3 \\
    0 
    \end{pmatrix} = \begin{pmatrix}
    -1 \\
    4 
    \end{pmatrix}
    ,
    \alpha \vec{a}_1 + \beta \vec{a}_2 = 
    \alpha \begin{pmatrix}
    1 \\
    2 
    \end{pmatrix} + \beta \begin{pmatrix}
    -1 \\
    0 
    \end{pmatrix} = \begin{pmatrix}
    \alpha \\
    2 \alpha
    \end{pmatrix} + \begin{pmatrix}
    -\beta \\
    0 
    \end{pmatrix}
    =\)

    \(
    = \begin{pmatrix}
    \alpha-\beta \\
    2\alpha
    \end{pmatrix};
    \)

    % 2 потрібно переробити, тому що там не вектор-стовпці, а вектор-рядки

    2) $\vec{b}_1 + \vec{b}_2 = $ \(
    \begin{pmatrix}
    1 \\
    -2 
    \end{pmatrix} +
    \begin{pmatrix}
    1 \\
    3 
    \end{pmatrix}
     = \begin{pmatrix}
    2 \\
    1 
    \end{pmatrix},
    \vec{b}_1 - \vec{b}_2 = \begin{pmatrix}
    1 \\
    -2 
    \end{pmatrix} -
    \begin{pmatrix}
    1 \\
    3 
    \end{pmatrix}
     = \begin{pmatrix}
    0 \\
    -5 
    \end{pmatrix}
    ,
    2\vec{b}_1 + 3\vec{b}_2 = 2\begin{pmatrix}
    1 \\
    -2 
    \end{pmatrix}+
    \)

    \(
    +3\begin{pmatrix}
    1 \\
    3 
    \end{pmatrix}
    =
    \begin{pmatrix}
    2 \\
    -4 
    \end{pmatrix} +
    \begin{pmatrix}
    3 \\
    9 
    \end{pmatrix} = \begin{pmatrix}
    5 \\
    5 
    \end{pmatrix}
    ,
    \alpha \vec{b}_1 + \beta \vec{b}_2 = 
    \alpha \begin{pmatrix}
    1 \\
    -2 
    \end{pmatrix} + \beta \begin{pmatrix}
    1 \\
    3 
    \end{pmatrix} = \begin{pmatrix}
    \alpha \\
    -2 \alpha
    \end{pmatrix} + \begin{pmatrix}
    \beta \\
    3\beta 
    \end{pmatrix}
    =\)

    \(
    = \begin{pmatrix}
    \alpha+\beta \\
    3\beta - 2\alpha
    \end{pmatrix};
    \)

    3) \(A + B = 
    \begin{pmatrix}
    1 & -1\\
    2 & 0 
    \end{pmatrix} + 
    \begin{pmatrix}
    1 & 1\\
    -2 & 3 
    \end{pmatrix} = 
    \begin{pmatrix}
    2 & 0\\
    0 & 3 
    \end{pmatrix}
    ,
    A - B = 
    \begin{pmatrix}
    1 & -1\\
    2 & 0 
    \end{pmatrix} - 
    \begin{pmatrix}
    1 & 1\\
    -2 & 3 
    \end{pmatrix} = 
    \begin{pmatrix}
    0 & -2\\
    4 & -3 
    \end{pmatrix},
    \)

    \(
    2A + 3B = 
    2\begin{pmatrix}
    1 & -1\\
    2 & 0 
    \end{pmatrix} + 
    3\begin{pmatrix}
    1 & 1\\
    -2 & 3 
    \end{pmatrix} = 
    \begin{pmatrix}
    2 & -2\\
    4 & 0 
    \end{pmatrix} + 
    \begin{pmatrix}
    3 & 3\\
    -6 & 9
    \end{pmatrix} = 
    \begin{pmatrix}
    5 & 1\\
    -2 & 9 
    \end{pmatrix},
    A - C
    \) ---
    
    неможливо провести додавання через різну розмірність,
    \(A - \lambda E_2 = 
    \begin{pmatrix}
    1 & -1\\
    2 & 0 
    \end{pmatrix} - 
    \lambda\begin{pmatrix}
    1 & 0\\
    0 & 1 
    \end{pmatrix} = 
    \begin{pmatrix}
    1 & -1\\
    2 & 0 
    \end{pmatrix} - 
    \begin{pmatrix}
    \lambda & 0\\
    0 & \lambda 
    \end{pmatrix} = 
    \begin{pmatrix}
    1-\lambda & -1\\
    2 & -\lambda 
    \end{pmatrix};
    \)

    4) \(C + D = 
    \begin{pmatrix}
    1 & 2\\
    4 & 0 \\
    5 & 7 
    \end{pmatrix} + 
    \begin{pmatrix}
    3 & -1 \\
    1 & 0 \\
    4 & -1 
    \end{pmatrix} = 
    \begin{pmatrix}
    4 & 1 \\
    5 & 0 \\
    9 & 6 
    \end{pmatrix}
    ,
    C - D = 
    \begin{pmatrix}
    1 & 2\\
    4 & 0 \\
    5 & 7 
    \end{pmatrix} - 
    \begin{pmatrix}
    3 & -1 \\
    1 & 0 \\
    4 & -1 
    \end{pmatrix} = 
    \begin{pmatrix}
    -2 & 3 \\
    3 & 0 \\
    1 & 8 
    \end{pmatrix},
    \)

    \(
    D - C= 
    \begin{pmatrix}
    3 & -1 \\
    1 & 0 \\
    4 & -1 
    \end{pmatrix} - 
    \begin{pmatrix}
    1 & 2\\
    4 & 0 \\
    5 & 7 
    \end{pmatrix} = 
    \begin{pmatrix}
    2 & -3\\
    -3 & 0 \\
    -1 & -8 
    \end{pmatrix},
    D - B
    \) ---  неможливо провести додавання через різну розмірність,
    \(B - \lambda E_2 = 
    \begin{pmatrix}
    1 & 1\\
    -2 & 3 
    \end{pmatrix} - 
    \lambda\begin{pmatrix}
    1 & 0\\
    0 & 1 
    \end{pmatrix} = 
    \begin{pmatrix}
    1 & 1\\
    -2 & 3 
    \end{pmatrix} - 
    \begin{pmatrix}
    \lambda & 0\\
    0 & \lambda 
    \end{pmatrix} = 
    \begin{pmatrix}
    1-\lambda & 1\\
    -2 & 3-\lambda 
    \end{pmatrix};
    \)

    5)








\end{document}