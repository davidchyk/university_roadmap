\documentclass[12pt,a4paper]{article}

\usepackage[utf8]{inputenc}
\usepackage[T2A]{fontenc}
\usepackage[ukrainian]{babel}
\usepackage{geometry}
\geometry{
    left=2cm,
    right=2cm,
    top=2cm,
    bottom=2cm,
    paperwidth=24.4cm,
    paperheight=28cm
}

\usepackage[table]{xcolor}
\usepackage{amsmath}
\usepackage{graphicx}
\usepackage{subcaption} % обов'язково в преамбулі
\usepackage{multirow}  % Для об'єднання рядків
\usepackage{makecell}         % для багато­рядкових комірок
\usepackage{pdflscape}       % або можна використати lscape

\begin{document}

    \setlength{\parindent}{0pt}

    \setcounter{page}{4}

    Таблиця станів регістрів:

    \vspace{1em}

    $M_Z = M_X \cdot M_Y = ,101010 \ \times \ ,100001:$

    \begin{table}[h!]

        \begin{tabular}{|c|c|c|c|c|p{9cm}|}
        \hline
        \textbf{№} &
        \makecell{\textbf{RG1} \\ \\ 7 \qquad \empty \quad 1} &    % <-- Одна комірка замість двох
        \makecell{\textbf{RG2} \\ \\ 6 \qquad \empty \quad 1} &
        \makecell{\textbf{RG3} \\ \\ 7 \qquad \empty \quad 1} &
        \makecell{\textbf{CT}  \\ \\ 3 \qquad \empty \quad 1} &
        \textbf{Мікрооперації} \\
        \hline
        \textbf{--} & 0000000 & 10101\textbf{0} & 0100001 & 110 &
        \makecell[l]{\texttt{RG1 = 0, RG2 = X, RG3 = 0.Y;}\\
        \texttt{CT = n = 6;}} \\ 
        \cline{1-3}
        \cline{5-6}
        \textbf{1} &
        \makecell{0000000\\[1em] \textit{Після зсуву:}\\ 0000000} &
        \makecell{101010\\[1em] \textit{Після зсуву:}\\ 01010\textbf{1}} & \empty & 101 &
        \makecell[l]{\texttt{RG1 = 0.r[RG1], RG2 = RG1[1].r[RG2],} \\
        \texttt{CT = CT - 1;}} \\
        \cline{1-3}
        \cline{5-6}
        \textbf{2} &
        \makecell[l]{
        \(
        \begin{array}{r} % r - вирівнюємо вправо для акуратності
        +0000000 \\
        \ \ 0100001 \\
        \hline
        0100001
        \end{array}
        \)
        \\[2em]
        \textit{Після зсуву:}\\
        0010000
        } &
        \makecell{010101\\[1em]
        \textit{Після зсуву:}\\ 10101\textbf{0}} &
        \empty &
        100 &
        \makecell[l]{\texttt{RG1 = RG1 + RG3;}\\
        \texttt{RG1 = 0.r[RG1], RG2 = RG1[1].r[RG2],}\\
        \texttt{CT = CT - 1;}} \\
        \cline{1-3}
        \cline{5-6}
        \textbf{3} &
        \makecell{0010000\\[1em] \textit{Після зсуву:}\\ 0001000} &
        \makecell{101010\\[1em] \textit{Після зсуву:}\\ 01010\textbf{1}} & \empty & 011 &
        \makecell[l]{\texttt{RG1 = 0.r[RG1], RG2 = RG1[1].r[RG2],} \\
        \texttt{CT = CT - 1;}} \\
        \cline{1-3}
        \cline{5-6}
        \textbf{4} &
        \makecell[l]{
        \(
        \begin{array}{r} % r - вирівнюємо вправо для акуратності
        +0001000 \\
        \ \ 0100001 \\
        \hline
        0101001
        \end{array}
        \)
        \\[2em]
        \textit{Після зсуву:}\\
        0010100
        } &
        \makecell{010101\\[1em]
        \textit{Після зсуву:}\\ 10101\textbf{0}} &
        \empty &
        010 &
        \makecell[l]{\texttt{RG1 = RG1 + RG3;}\\
        \texttt{RG1 = 0.r[RG1], RG2 = RG1[1].r[RG2],}\\
        \texttt{CT = CT - 1;}} \\
        \cline{1-3}
        \cline{5-6}
        \textbf{5} &
        \makecell{0010100\\[1em] \textit{Після зсуву:}\\ 0001010} &
        \makecell{101010\\[1em] \textit{Після зсуву:}\\ 01010\textbf{1}} & \empty & 001 &
        \makecell[l]{\texttt{RG1 = 0.r[RG1], RG2 = RG1[1].r[RG2],} \\
        \texttt{CT = CT - 1;}} \\
        \cline{1-3}
        \cline{5-6}
        \textbf{6} &
        \makecell[l]{
        \(
        \begin{array}{r} % r - вирівнюємо вправо для акуратності
        +0001010 \\
        \ \ 0100001 \\
        \hline
        0101011
        \end{array}
        \)
        \\[2em]
        \textit{Після зсуву:}\\
        0010101
        } &
        \makecell{010101\\[1em]
        \textit{Після зсуву:}\\ 101010} &
        \empty &
        \textbf{000} &
        \makecell[l]{\texttt{RG1 = RG1 + RG3;}\\
        \texttt{RG1 = 0.r[RG1], RG2 = RG1[1].r[RG2],}\\
        \texttt{CT = CT - 1;}} \\
        \hline

        \end{tabular}

    \end{table}

    Знаковий біт мантиси $M_Z: 1 \oplus 0 = 1$ (число від'ємне). Тобто $M_Z = 1,010101101010$, а $P_Z = P_X + P_Y = 4_{10} + 3_{10} = 7_{10} = 0111_2$.
    Але старший біт (без знакового) мантиси $M_Z$ не дорвінює 1, тому нормалізуємо, і отримаємо:

    $M_Z = 1,10101101010$, $P_Z = 0110$.

    Також звернемо увагу, що потрібно виділити 6 бітів (без знакового) в мантисі, тому потрібно виконати округлення до 6 розрядів. Помітимо, що 7 біт мантиси = 0, це значить, що округлення можна провести шляхом
    відкидання бітів, окрім перших 6. Тобто, $M_Z \approx 1,101011$, $P_Z = 0110$. Звідси
    \begin{flalign*}
        &
        \begin{array}{r@{\ =\ }c c c c}
        Z & \fbox{0}     & \fbox{0110} & \fbox{1} & \fbox{101011}
        \end{array}
        &
    \end{flalign*}

\end{document}