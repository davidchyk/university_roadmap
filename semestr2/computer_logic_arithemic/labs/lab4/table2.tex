\documentclass[12pt,a4paper]{article}

\usepackage[utf8]{inputenc}
\usepackage[T2A]{fontenc}
\usepackage[ukrainian]{babel}
\usepackage{geometry}
\geometry{
    left=2cm,
    right=2cm,
    top=2cm,
    bottom=2cm,
    paperwidth=24.4cm,
    paperheight=28.5cm
}

\usepackage[table]{xcolor}
\usepackage{amsmath}
\usepackage{graphicx}
\usepackage{subcaption} % обов'язково в преамбулі
\usepackage{multirow}  % Для об'єднання рядків
\usepackage{makecell}         % для багато­рядкових комірок
\usepackage{pdflscape}       % або можна використати lscape

\begin{document}

    \setlength{\parindent}{0pt}

    \setcounter{page}{9}

    Таблиця станів регістрів:

    \vspace{1em}

    $M_N = M_Z \cdot M_Y = ,101011 \ \times \ ,100001:$

    \begin{table}[h!]

        \begin{tabular}{|c|c|c|c|c|p{9cm}|}
        \hline
        \textbf{№} &
        \makecell{\textbf{RG2} \\ \\ 7 \qquad \empty \quad 1} &    % <-- Одна комірка замість двох
        \makecell{\textbf{RG1} \\ \\ 6 \qquad \empty \quad 1} &
        \makecell{\textbf{RG3} \\ \\ 6 \qquad \empty \quad 1} &
        \makecell{\textbf{CT}  \\ \\ 3 \qquad \empty \quad 1} &
        \textbf{Мікрооперації} \\
        \hline
        \textbf{--} & \textbf{1}010110 & 000000 & 100001 & 110 &
        \makecell[l]{\texttt{RG3 := 0.Y, RG2 := X.0, RG1 := 0;}\\
        \texttt{CT := n := 6; CT = CT - 1;}} \\ 
        \cline{1-3}
        \cline{5-6}
        \textbf{1} &
        \makecell{1010110\\[1em] \textit{Після зсуву:}\\ \textbf{0}101101} &
        \makecell[l]{
        \(
        \begin{array}{r} % r - вирівнюємо вправо для акуратності
        +000000 \\
        \ \ 100001 \\
        \hline
        100001
        \end{array}
        \)
        \\[2em]
        \textit{Після зсуву:}\\
        000010
        } 
        & \empty & 101 &
        \makecell[l]{\texttt{RG1 := RG1 + RG3, RG2 := RG2 + 0 + CI;} \\
        \texttt{RG1 := l(RG1).0, RG2 := l(RG2).RG1[6];} \\
        \texttt{CT = CT - 1;}} \\
        \cline{1-3}
        \cline{5-6}
        \textbf{2} &
        \makecell{0101101\\[1em]
        \textit{Після зсуву:}\\ \textbf{1}011010} &
        \makecell{000010\\[1em]
        \textit{Після зсуву:}\\ 000100} &
        \empty &
        100 &
        \makecell[l]{\texttt{RG1 := l(RG1).0, RG2 := l(RG2).RG1[6];}\\
        \texttt{CT = CT - 1;}} \\
        \cline{1-3}
        \cline{5-6}
        \textbf{3} &
        \makecell{1011010\\[1em] \textit{Після зсуву:}\\ \textbf{0}110101} &
        \makecell[l]{
        \(
        \begin{array}{r} % r - вирівнюємо вправо для акуратності
        +000100 \\
        \ \ 100001 \\
        \hline
        100101
        \end{array}
        \)
        \\[2em]
        \textit{Після зсуву:}\\
        001010
        } &
        \empty & 011 &
        \makecell[l]{\texttt{RG1 := RG1 + RG3, RG2 := RG2 + 0 + CI;} \\
        \texttt{RG1 := l(RG1).0, RG2 := l(RG2).RG1[6];} \\
        \texttt{CT = CT - 1;}} \\
        \cline{1-3}
        \cline{5-6}
        \textbf{4} &
        \makecell{0110101\\[1em]
        \textit{Після зсуву:}\\ \textbf{1}101010} &
        \makecell{001010\\[1em]
        \textit{Після зсуву:}\\ 010100} &
        \empty &
        010 &
        \makecell[l]{\texttt{RG1 := l(RG1).0, RG2 := l(RG2).RG1[6];}\\
        \texttt{CT = CT - 1;}} \\
        \cline{1-3}
        \cline{5-6}
        \textbf{5} &
        \makecell{1101010\\[1em] \textit{Після зсуву:}\\ \textbf{1}010101} &
        \makecell[l]{
        \(
        \begin{array}{r} % r - вирівнюємо вправо для акуратності
        +010100 \\
        \ \ 100001 \\
        \hline
        110101
        \end{array}
        \)
        \\[2em]
        \textit{Після зсуву:}\\
        101010
        } &
        \empty & 001 &
        \makecell[l]{\texttt{RG1 := RG1 + RG3, RG2 := RG2 + 0 + CI;} \\
        \texttt{RG1 := l(RG1).0, RG2 := l(RG2).RG1[6];} \\
        \texttt{CT = CT - 1;}} \\
        \cline{1-3}
        \cline{5-6}
        \textbf{6} &
        \makecell[l]{
        \(
        \begin{array}{r} % r - вирівнюємо вправо для акуратності
        +1010101 \\
        \ \ 0000001 \\
        \hline
        1010110
        \end{array}
        \)
        \\[2em]
        \textit{Після зсуву:}\\
        0101100
        } &
        \makecell[l]{
        \(
        \begin{array}{r} % r - вирівнюємо вправо для акуратності
        +101010 \\
        \ \ 100001 \\
        \hline
        001011
        \end{array}
        \)
        \\[2em]
        \textit{Після зсуву:}\\
        010110
        } &
        \empty & \textbf{000} &
        \makecell[l]{\texttt{RG1 := RG1 + RG3, RG2 := RG2 + 0 + CI;} \\
        \texttt{RG1 := l(RG1).0, RG2 := l(RG2).RG1[6];} \\
        \texttt{CT = CT - 1;}} \\
        \cline{1-3}
        \cline{5-6}
        \hline

        \end{tabular}

    \end{table}

    Знаковий біт мантиси $M_N: 1 \oplus 0 = 1$ (число від'ємне). Тобто $M_N = 1,010110001011$, а $P_N = P_Z + P_Y = 6_{10} + 3_{10} = 9_{10} = 1001_2$.
    Але старший біт (без знакового) мантиси $M_N$ не дорвінює 1, тому нормалізуємо, і отримаємо:

    $M_N = 1,101100010110$, $P_N = 1000$.

    Також звернемо увагу, що потрібно виділити 6 бітів (без знакового) в мантисі, тому потрібно виконати округлення до 6 розрядів. Помітимо, що 7 біт мантиси = 0, це значить, що округлення можна провести шляхом
    відкидання бітів, окрім перших 6. Тобто, $M_N \approx 1,101100$, $P_N = 1000$. Звідси
    \begin{flalign*}
        &
        \begin{array}{r@{\ =\ }c c c c}
        N & \fbox{0}     & \fbox{1000} & \fbox{1} & \fbox{101100}
        \end{array}
        &
    \end{flalign*}

\end{document}