\documentclass[12pt,a4paper]{article}

\usepackage[utf8]{inputenc}
\usepackage[T2A]{fontenc}
\usepackage[ukrainian]{babel}
\usepackage{geometry}
\geometry{
    left=2cm,
    right=2cm,
    top=2cm,
    bottom=2cm,
    paperwidth=28cm,
    paperheight=40cm
}

\usepackage[table]{xcolor}
\usepackage{amsmath}
\usepackage{graphicx}
\usepackage{subcaption} % обов'язково в преамбулі
\usepackage{multirow}  % Для об'єднання рядків
\usepackage{makecell}         % для багато­рядкових комірок
\usepackage{pdflscape}       % або можна використати lscape

\begin{document}

    \setlength{\parindent}{0pt}

    \setcounter{page}{14}

    Таблиця станів регістрів:

    \vspace{1em}

    $M_F = M_N \slash M_Y = ,101100 \ \slash \ ,100001$. Звідси бачимо, що умова $M_N < M_Y$ НЕ виконується, тому
    зсунемо мантису $M_N$ на 1 біт вправо, і додамо 1 до порядку $P_N$. Тим самим, отримуємо $M_N = ,010110$, $P_N = 0111$. Тут же виконується умова $M_N < M_Y$.

    \begin{table}[h!]

        \begin{tabular}{|c|c|c|c|p{9cm}|}
        \hline
        \textbf{№} &
        \makecell{\textbf{RG3} \\ \\ 7 \qquad \empty \quad 1} &    % <-- Одна комірка замість двох
        \makecell{\textbf{RG2} \\ \\ 8 \qquad \empty \quad 1} &
        \makecell{\textbf{RG1} \\ \\ 8 \qquad \empty \quad 1} &
        \textbf{Мікрооперації} \\
        \hline
        \textbf{--} & 1111111 & \textbf{0}0010110 &
        \makecell[l]{
        $RG1$ = 00100001\\[1em]
        $-RG1 + 1$ (від'ємне число $00.Y$ в ДК):\\[1em]
        \(
        \begin{array}{r} % r - вирівнюємо вправо для акуратності
        +11011110 \\
        \ \ 00000001 \\
        \hline
        11011111
        \end{array}
        \)
        } &
        \makecell[l]{\texttt{RG3 = 1..1; RG2 = 00.X; RG1 = 00.Y;}}\\
        \cline{1-3}
        \cline{5-5}
        \textbf{1} &
        \makecell{1111111\\[1em] \textit{Після зсуву:}\\ 111111\textcolor{red}{0}} &
        \makecell[l]{
        \(
        \begin{array}{r} % r - вирівнюємо вправо для акуратності
        +00010110 \\
        \ \ 11011111 \\
        \hline
        \textcolor{red}{1}1110101
        \end{array}
        \)
        \\[2em]
        \textit{Після зсуву:}\\
        \textbf{1}1101010
        } 
        & \empty &
        \makecell[l]{\texttt{RG2 = RG2 + (-RG1) + D;} \\
        \texttt{RG3 = l(RG3).(!RG2[8]); RG2 = l(RG2).0;}} \\
        \cline{1-3}
        \cline{5-5}
        \textbf{2} &
        \makecell{1111110\\[1em]
        \textit{Після зсуву:}\\ 111110\textcolor{red}{1}} &
        \makecell[l]{
        \(
        \begin{array}{r} % r - вирівнюємо вправо для акуратності
        +11101010 \\
        \ \ 00100001 \\
        \hline
        \textcolor{red}{0}0001011
        \end{array}
        \)
        \\[2em]
        \textit{Після зсуву:}\\
        \textbf{0}0010110
        } &
        \empty &
        \makecell[l]{\texttt{RG2 = RG2 + RG1;}\\
        \texttt{RG3 = l(RG3).(!RG2[8]); RG2 = l(RG2).0;}} \\
        \cline{1-3}
        \cline{5-5}
        \textbf{3} &
        \makecell{1111101\\[1em] \textit{Після зсуву:}\\ 111101\textcolor{red}{0}} &
        \makecell[l]{
        \(
        \begin{array}{r} % r - вирівнюємо вправо для акуратності
        +00010110 \\
        \ \ 11011111 \\
        \hline
        \textcolor{red}{1}1110101
        \end{array}
        \)
        \\[2em]
        \textit{Після зсуву:}\\
        \textbf{1}1101010
        } 
        & \empty &
        \makecell[l]{\texttt{RG2 = RG2 + (-RG1) + D;} \\
        \texttt{RG3 = l(RG3).(!RG2[8]); RG2 = l(RG2).0;}} \\
        \cline{1-3}
        \cline{5-5}
        \textbf{4} &
        \makecell{1111010\\[1em]
        \textit{Після зсуву:}\\ 111010\textcolor{red}{1}} &
        \makecell[l]{
        \(
        \begin{array}{r} % r - вирівнюємо вправо для акуратності
        +11101010 \\
        \ \ 00100001 \\
        \hline
        \textcolor{red}{0}0001011
        \end{array}
        \)
        \\[2em]
        \textit{Після зсуву:}\\
        \textbf{0}0010110
        } &
        \empty &
        \makecell[l]{\texttt{RG2 = RG2 + RG1;}\\
        \texttt{RG3 = l(RG3).(!RG2[8]); RG2 = l(RG2).0;}} \\
        \cline{1-3}
        \cline{5-5}
        \textbf{5} &
        \makecell{1110101\\[1em] \textit{Після зсуву:}\\ 110101\textcolor{red}{0}} &
        \makecell[l]{
        \(
        \begin{array}{r} % r - вирівнюємо вправо для акуратності
        +00010110 \\
        \ \ 11011111 \\
        \hline
        \textcolor{red}{1}1110101
        \end{array}
        \)
        \\[2em]
        \textit{Після зсуву:}\\
        \textbf{1}1101010
        } 
        & \empty &
        \makecell[l]{\texttt{RG2 = RG2 + (-RG1) + D;} \\
        \texttt{RG3 = l(RG3).(!RG2[8]); RG2 = l(RG2).0;}} \\
        \cline{1-3}
        \cline{5-5}
        \textbf{6} &
        \makecell{1101010\\[1em]
        \textit{Після зсуву:}\\ 101010\textcolor{red}{1}} &
        \makecell[l]{
        \(
        \begin{array}{r} % r - вирівнюємо вправо для акуратності
        +11101010 \\
        \ \ 00100001 \\
        \hline
        \textcolor{red}{0}0001011
        \end{array}
        \)
        \\[2em]
        \textit{Після зсуву:}\\
        \textbf{0}0010110
        } &
        \empty &
        \makecell[l]{\texttt{RG2 = RG2 + RG1;}\\
        \texttt{RG3 = l(RG3).(!RG2[8]); RG2 = l(RG2).0;}} \\
        \cline{1-3}
        \cline{5-5}
        \textbf{7} &
        \makecell{1010101\\[1em] \textit{Після зсуву:}\\ 010101\textcolor{red}{0}} &
        \makecell[l]{
        \(
        \begin{array}{r} % r - вирівнюємо вправо для акуратності
        +00010110 \\
        \ \ 11011111 \\
        \hline
        \textcolor{red}{1}1110101
        \end{array}
        \)
        \\[2em]
        \textit{Після зсуву:}\\
        11101010
        } 
        & \empty &
        \makecell[l]{\texttt{RG2 = RG2 + (-RG1) + D;} \\
        \texttt{RG3 = l(RG3).(!RG2[8]); RG2 = l(RG2).0;}} \\
        \cline{1-3}
        \cline{5-5}
        \hline

        \end{tabular}

    \end{table}

    Знаковий біт мантиси $M_F: 1 \oplus 0 = 1$ (число від'ємне). Тобто $M_F = 1,10101011101010$, а $P_F = P_N + P_Y = 7_{10} + 3_{10} = 10_{10} = 1010_2$.
    Як бачимо, старший біт мантиси є 1, тому нормалізація не потрібна. Тобто:

    $M_F = 1,10101011101010$, $P_F = 1010$.

    Також звернемо увагу, що потрібно виділити 6 бітів (без знакового) в мантисі, тому потрібно виконати округлення до 6 розрядів. Помітимо, що 7 біт мантиси = 1, це значить, що округлення можна провести шляхом
    додавання одиниці до 7 біту із подальшим переносом одиниць. Тобто,

    \begin{flalign*}
        &
        \begin{array}{r}
        M_F: 1010101 \\
        + \phantom{1\ }0000001 \\
        \hline
        1010110
        \end{array}
        &
    \end{flalign*}
    
    Звідси $M_F \approx 1,101011$, $P_F = 1010$, Або
    \begin{flalign*}
        &
        \begin{array}{r@{\ =\ }c c c c}
        F & \fbox{0}     & \fbox{1010} & \fbox{1} & \fbox{101011}
        \end{array}
        &
    \end{flalign*}

\end{document}