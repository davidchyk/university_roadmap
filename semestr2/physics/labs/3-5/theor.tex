\documentclass[12pt,a4paper]{article}
\usepackage[utf8]{inputenc}
\usepackage[T2A]{fontenc}
\usepackage[ukrainian]{babel}
\usepackage{fancyvrb}
\usepackage{pdflscape}

\usepackage{amsmath} % у преамбулі
\usepackage{array, multirow}
\usepackage{hyperref} % <-- Обов'язково підключіть цей пакет
\usepackage{caption}
\usepackage{booktabs}
\usepackage{subcaption} % для підписів (а), (б)
\usepackage{breqn} % Пакет для автоматичного перенесення виразів
\usepackage{mathtools} % Для додаткових можливостей, наприклад, для створення кастомних конструкцій
\usepackage{makecell} % Для створення багаторядкових комірок у таблицях
\usepackage{enumitem}

\usepackage{xcolor}

\renewcommand{\thetable}{№\arabic{table}}
\captionsetup[table]{name=Таблиця}  % замість "Табл." буде "Таблиця"

\usepackage{graphicx} % <-- Для роботи з \includegraphics
\usepackage{geometry}
\geometry{
    left=2cm,
    right=2cm,
    top=2cm,
    bottom=2cm
}


\begin{document}

    \begin{titlepage}

        \thispagestyle{empty}
        \begin{center}
        \large
        Національний технічний університет України\\
        «Київський політехнічний інститут імені Ігоря Сікорського»\\[1em]
        Факультет інформатики та обчислювальної техніки\\
        Кафедра загальної фізики
        \end{center}

        \vfill

        \begin{center}
        \textbf{\LARGE Фізика}\\[2em]
        \textbf{\Large Лабораторна робота №3-5}\\
        «Вивчення поляризованого світла» 
        \end{center}

        \vfill

        \begin{flushright}
        Виконав: студент 1 курсу ФІОТ, гр. ІО-41\\
        \textit{Давидчук А. М.}\\
        Залікова книжка № 4106\\[1em]
        Перевірив: \textit{Колган В.\,В.}
        \end{flushright}

        \vfill

        \begin{center}
        Київ -- 2025
        \end{center}

    \end{titlepage}

    \setlength{\parindent}{0pt}

    \textbf{\underline{Тема:}} «Вивчення поляризованого світла».

    \vspace{1em}

    \textbf{\underline{Мета:}} експериментальне перевірити формули Френеля, досліджуючи
    відбивання поляризованого світла від скляної пластинки, та визначити кут
    Брюстера, показник заломлення скла та площину коливань світлового вектора $\vec{E}$.

    \vspace{1.5em}

    \begin{center} \textbf{\large Теоретичні відомості} \end{center}
    \setlength{\parindent}{1.5em}

    \begin{center} \textbf{Природне і поляризоване світло. Поляризатори} \end{center}

    Як відомо, світло являє собою поперечну електромагнітну хвилю. Світлові хвилі
    бувають природними та поляризованими, тобто такими, в яких (на відміну від
    природних) коливання вектора $\vec{E}$ певним чином упорядковані. Способи
    впорядкування, а у відповідності до них і види поляризації. Оптичні пристрої, за
    допомогою яких світло поляризується, називаються поляризаторами.

    \begin{center} \textbf{Відбивання плоскої лінійно поляризованої хвилі від діелектричної пластинки.} \end{center}

    При розгляді цього питання хвилю, що падає, представляють у вигляді
    суперпозиції двох хвиль \(\vec{E}_{\parallel 0}\) та \(\vec{E}_{\perp 0}\), електричні вектори яких коливаються відповідно у
    площині падіння хвилі та перпендикулярно до неї (див. рис. 5.1).

    \begin{figure}[!ht]

        \renewcommand{\thefigure}{5.\arabic{figure}} % робимо "3.1", "3.2" і т.д.

        \centering
        % Підставляєте потрібний шлях та розмір зображення:
        \includegraphics[width=0.4\textwidth]{5.1.png}
        % Підпис (зазвичай під малюнком):
        \caption{}
        % Мітка для посилань у тексті (\ref{fig:...})
        \label{fig1:schema}

    \end{figure}

    Залежність амплітуди
    відбитої й заломленої хвиль від кута падіння описується формулами Френеля.

    Тут $n_1, n_2$ --- абсолютні показники заломлення повітря й скла;
    $\vartheta_1, \vartheta_2$ --- кути падіння і заломлення хвилі.

    Так, наприклад, амплітуди відбитих хвиль \(\vec{E}_{\parallel 0}\) та \(\vec{E}_{\perp 0}\) відповідно до цих формул

    \begin{equation}
        E_{\parallel}
        =E_{\parallel0}\,\frac{\tg(\vartheta_1-\vartheta_2)}{\tg(\vartheta_1+\vartheta_2)},
        \quad
        E_{\perp}
        =E_{\perp0}\,\frac{\sin(\vartheta_1-\vartheta_2)}{\sin(\vartheta_1+\vartheta_2)}.
        \tag{5.1}
    \end{equation}

    по різному залежать від кута падіння $\vartheta_1$.

    З формул Френеля (5.1) видно, що за умови $\vartheta_1 = \vartheta_2 = \pi \slash 2$
    амплітуда, відбитої хвилі $E_{\parallel}$ стає рівною нулю, і відбите світло містить лише компонент
    $E_{\perp}$, тобто воно є повністю
    поляризованим. Величина кута падіння, при якому це відбувається, визначається з
    умови $\tg \vartheta_{\text{Бр}} = n_2 / n_1$. Ця умова носить назву умови Брюстера, або ж закону Брюстера.

    Оскільки кути $\vartheta_1, \vartheta_2$, які фігурують у (5.1), пов'язані законом заломлення світла
    $\left( \sin \vartheta_1 \right.$ $\left. \slash \sin \vartheta_2 = n_2 / n_1 \right)$
    кут $\vartheta_2$ можна виразити через $\vartheta_1$ і, таким чином, одержати функцію,
    яка описує залежність амплітуди відбитих хвиль від кута падіння
    $\vartheta_1$.

    \newpage

    На рис. 5.2 показані графіки функції $E_{\parallel} / E_0 = f(\vartheta_1)$ (крива I)
    та $E_{\perp} / E_0 = f(\vartheta_1)$ (крива II),
    розраховані для випадку, коли $n_1 = 1, n_2 = 1{,}5$.

    \begin{figure}[!ht]

        \renewcommand{\thefigure}{5.\arabic{figure}} % робимо "3.1", "3.2" і т.д.

        \centering
        % Підставляєте потрібний шлях та розмір зображення:
        \includegraphics[width=0.4\textwidth]{5.2.png}
        % Підпис (зазвичай під малюнком):
        \caption{}
        % Мітка для посилань у тексті (\ref{fig:...})
        \label{fig2:schema}

    \end{figure}

    Як з рис. 5.2, криві залежностей для $\perp$ та $\parallel$ поляризацій вектора напруженості $\vec{E}$
    суттєво різняться, що дозволяє за результатами експерименту встановити площину
    поляризації хвилі, яка падає на скло, величину кута Брюстера та показник заломлення
    скла.

    \begin{center} \textbf{Проходження лінійно поляризованої хвилі через поляризатор. Закон Малюса} \end{center}

    Якщо лінійно поляризована світлова хвиля падає нормально на поляризатор так,
    що площина коливань її вектора $\vec{E}$ складає з головною оптичною площиною
    поляризатора (11) кут $\alpha$ (рис. 5.N),
    то інтенсивність $I$ хвилі, що пройшла, визначається
    виразом

    \begin{equation}
        I = I_0 \cos^2 \alpha,
        \tag{5.2}
    \end{equation}

    де $I_0$ --- інтенсивність світла, що падає. Це співвідношення називається законом Малюса.
    Знаючи площину поляризатора та оцінюючи інтенсивність світла, що пройшло, можна
    за законом Малюса встановити площину коливань досліджуваного лінійно
    поляризованого світла.

    \begin{center} \textbf{\large Практична частина} \end{center}

    \begin{center} \textbf{Методика вимірювання інтенсивності та амплітуди світлової хвилі} \end{center}

    Під інтенсивністю світла розуміють усереднену величину модуля
    густини потоку енергії світлової хвилі

    \begin{equation}
        I = \left\langle \left| \vec{S} \right| \right\rangle = \frac{1}{2} \sqrt{\frac{\varepsilon_0}{\mu_0}}E_m^2,
        \tag{5.3}
    \end{equation}

    де $\varepsilon_0, \mu_0$ --- відповідно електрична і магнітна сталі,
    $E_m$ --- амплітуда світлової хвилі.

    Інтенсивність і амплітуда світлової хвилі у цій роботі вимірюються за допомогою
    приймача випромінювання, в якому використовується вентильний фотоефект.

    Приймач складається з фотоелектричного датчика, що перетворює світловий
    потік у фотоЕРС, і вольтметра для вимірювання останньої.

    Пропорційність між фотоЕРС та інтенсивністю світлової хвилі (за малих
    інтенсивностей) забезпечується законами внутрішнього фотоефекту.

    Таким чином, інтенсивність світлової хвилі виявляється пропорційною показам
    вольтметра $I \sim U$, а її амплітуда --- кореню квадратному з показань приладу
    $E_m \sim \sqrt{U}$.

    \textbf{Визначення виду поляризації світлової хвилі}

    Для прикладу розглянемо методику ідентифікації лише лінійно поляризованої хвилі.

    Лінійно поляризована хвиля легко пізнається, якщо пропускати її крізь
    поляризатор. Як відмічалося раніше, інтенсивність хвилі, що пройшла, у цьому випадку
    підпорядкована закону Малюса: $I = I_0 \cos^2 \alpha$.

    Обертаючи аналізатор у площині,
    нормальній до напрямку поширення хвилі, можна знайти два його характерні
    положення: у першому інтенсивність світла , що пройшла, максимальна, у другому
    (відрізняється на $90 \circ$ від першого) --- нульова. Для більшої переконливості закон (5.2)
    може бути перевіреним у повному об'ємі.

    \begin{center} \textbf{Опис експериментальної установки} \end{center}

    Основною деталлю експериментальної
    установки є вимірювальна головка з оптичними
    елементами та лімбом 1 (рис. 5.3).

    \begin{figure}[!ht]

        \renewcommand{\thefigure}{5.\arabic{figure}} % робимо "3.1", "3.2" і т.д.

        \centering
        % Підставляєте потрібний шлях та розмір зображення:
        \includegraphics[width=0.4\textwidth]{5.3.png}
        % Підпис (зазвичай під малюнком):
        \caption{}
        % Мітка для посилань у тексті (\ref{fig:...})
        \label{fig3:schema}

    \end{figure}

    Головка може бути встановлена у двох положеннях:

    \begin{enumerate}[label=\alph*)]
        \item вертикально для зняття залежності амплітуди відбитої хвилі від кута падіння;
        \item горизонтально для перевірки закону Малюса і виду поляризації світлової хвилі.
    \end{enumerate}

    У верхній частині головки встановлені
    плоскопаралельна пластинка 2, фотоприймач 3,
    екран 4. У нижній її частині --- поляроїд 6 та
    другий фотоприймач 5. Фотоприймачі з'єднані з
    вольтметром 10.

    Джерелом поляризованого світла є He-Ne
    лазер. У ньому є джерело живлення 7, газорозрядна трубка 8 яка знаходиться між
    дзеркалами резонатора 9.

    Довжина хвилі лазерного випромінювання $\lambda = 0{,}63$ мкм,
    розходження пучка $30\circ$ потужність $\sim 1$ МВт.

    На передньому торці лазера намальовані взаємно перпендикулярні лінії І і II.
    Уздовж однієї з них відбувається коливання світлового вектора $\vec{E}$ . В установці
    передбачена можливість зміни напрямку коливань світлового вектора відносно
    діелектричної пластинки (скло) шляхом обертання лазера навколо своєї осі.

    Увага! Попадання в очі прямого лазерного пучка небезпечно для зору! Світло лазера
    можна спостерігати тільки після відбиття від поверхонь, що розсіюють.

    \newpage

    \begin{center} \textbf{Порядок виконання} \end{center}

    \newpage

    start

\end{document}